\documentclass{tufte-handout}

%\geometry{showframe}% for debugging purposes -- displays the margins

\usepackage{amsmath}

% Set up the images/graphics package
\usepackage{graphicx}
\setkeys{Gin}{width=\linewidth,totalheight=\textheight,keepaspectratio}
\graphicspath{{graphics/}}

\title{Analysis of RPGLite data}
\author[]{Tom Wallis \& William Kavanagh}
\date{April 2020}  % if the \date{} command is left out, the current date will be used

% The following package makes prettier tables.  We're all about the bling!
\usepackage{booktabs}

% The units package provides nice, non-stacked fractions and better spacing
% for units.
\usepackage{units}

% The fancyvrb package lets us customize the formatting of verbatim
% environments.  We use a slightly smaller font.
\usepackage{fancyvrb}
\fvset{fontsize=\normalsize}

% Small sections of multiple columns
\usepackage{multicol}

% Provides paragraphs of dummy text
\usepackage{lipsum}

% These commands are used to pretty-print LaTeX commands
\newcommand{\doccmd}[1]{\texttt{\textbackslash#1}}% command name -- adds backslash automatically
\newcommand{\docopt}[1]{\ensuremath{\langle}\textrm{\textit{#1}}\ensuremath{\rangle}}% optional command argument
\newcommand{\docarg}[1]{\textrm{\textit{#1}}}% (required) command argument
\newenvironment{docspec}{\begin{quote}\noindent}{\end{quote}}% command specification environment
\newcommand{\docenv}[1]{\textsf{#1}}% environment name
\newcommand{\docpkg}[1]{\texttt{#1}}% package name
\newcommand{\doccls}[1]{\texttt{#1}}% document class name
\newcommand{\docclsopt}[1]{\texttt{#1}}% document class option name

\begin{document}

\maketitle% this prints the handout title, author, and date

\begin{abstract}
\noindent This document describes the data generated by the RPGLite experiment, the analysis we have performed on it thus far and areas where we should like to perform further analysis
\end{abstract}

%\printclassoptions

\section{The Data}

Data is collected in as part of a Mongo database and organised into six collections of similar documents detailed here with their notable fields:
\begin{enumerate}
    \item \textbf{completed\_games}: 
    \begin{itemize}
        \item usernames -- an array of the usernames for players 1 and 2 respectively,
        \item start\_time -- python datetime object given moment of game creation to millisecond precision,
        \item Moves -- list of moves made in notation of the form $pxC^1pyC^2\_nn$ which would denote player $x \in \{1,2\}$ using character $C^1 \in \{K,A,R,H,W,B,M,G\}$  to attack player $y$'s character $C^2$ and rolling $nn \in \{0..99\}$ 
    \end{itemize}
    \item \textbf{games}:
    \item \textbf{page\_hits}:
    \item \textbf{player\_backup}
    \item \textbf{players}:
    \item \textbf{special\_data}:
\end{enumerate}

\footnote{this may not be a necessary section to include, though it may be useful if this document is to be used as a reference for \textit{outsiders}.}

\newpage



\end{document}
