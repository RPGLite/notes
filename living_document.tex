\documentclass{tufte-handout}

%\geometry{showframe}% for debugging purposes -- displays the margins

\usepackage{amsmath}

% Set up the images/graphics package
\usepackage{graphicx}
\setkeys{Gin}{width=\linewidth,totalheight=\textheight,keepaspectratio}
\graphicspath{{graphics/}}

\title{Analysis of RPGLite data}
\author[]{Tom Wallis \& William Kavanagh}
\date{April 2020}  % if the \date{} command is left out, the current date will be used

\usepackage{hyperref}
\hypersetup{
    colorlinks=true,
    linkcolor=blue,
    filecolor=magenta,      
    urlcolor=cyan,
}

% The following package makes prettier tables.  We're all about the bling!
\usepackage{booktabs}

% The units package provides nice, non-stacked fractions and better spacing
% for units.
\usepackage{units}

% The fancyvrb package lets us customize the formatting of verbatim
% environments.  We use a slightly smaller font.
\usepackage{fancyvrb}
\fvset{fontsize=\normalsize}

% Small sections of multiple columns
\usepackage{multicol}

% Provides paragraphs of dummy text
\usepackage{lipsum}

\usepackage[obeyFinal,textsize=footnotesize]{todonotes}

% These commands are used to pretty-print LaTeX commands
\newcommand{\doccmd}[1]{\texttt{\textbackslash#1}}% command name -- adds backslash automatically
\newcommand{\docopt}[1]{\ensuremath{\langle}\textrm{\textit{#1}}\ensuremath{\rangle}}% optional command argument
\newcommand{\docarg}[1]{\textrm{\textit{#1}}}% (required) command argument
\newenvironment{docspec}{\begin{quote}\noindent}{\end{quote}}% command specification environment
\newcommand{\docenv}[1]{\textsf{#1}}% environment name
\newcommand{\docpkg}[1]{\texttt{#1}}% package name
\newcommand{\doccls}[1]{\texttt{#1}}% document class name
\newcommand{\docclsopt}[1]{\texttt{#1}}% document class option name

\newcommand{\WilliamToDo}[1]{\textcolor{blue}{\textbf{W:~{#1}}}}

\begin{document}

\maketitle% this prints the handout title, author, and date

\begin{abstract}
\noindent This document describes the data generated by the RPGLite experiment, the analysis we have performed on it thus far and areas where we should like to perform further analysis
\end{abstract}

%\printclassoptions

\section{The Data}

\todo{Hey, William! I imagine I'll be editing this sometimes from the github repository. After you do chunks of work on this could you hit the menu above and make sure it's synced to github? We have a repo for this now, rpglite/notes I think. \textcolor{blue}{Of course I can mate.}}


Data\footnote{this may not be a necessary section to include, though it may be useful if this document is to be used as a reference for \textit{outsiders}.} is collected in the form of a Mongo database and organised into six collections of similar documents detailed here with their notable fields:
\begin{enumerate}
    \item \textbf{games}: in progress games
    \begin{itemize}
        \item usernames -- an array of the usernames for players 1 and 2 respectively,
        \item start\_time -- python datetime object given moment of game creation to millisecond precision,
        \item Moves -- list of moves made in notation of the form $pxC^1pyC^2\_nn$ which would denote player $x \in \{1,2\}$ using character $C^1 \in \{K,A,R,H,W,B,M,G\}$  to attack player $y$'s character $C^2$ and rolling $nn \in \{0..99\}$ 
    \end{itemize}
    \item \textbf{completed\_games}: games now completed
        \begin{itemize}
        \item as above.
        \item winner -- {1,2} denoting winner if one was found (can be non present if game was abandon)
    \end{itemize}
    \item \textbf{page\_hits}:
    \item \textbf{player\_backup}
    \item \textbf{players}:
    \item \textbf{special\_data}:
\end{enumerate}

\section{Initial Analysis}

\subsection{win-delta / pick-rate}\footnote{The three figures described here are all generated by running \textit{rolling\_analysis.py}}

A common form for game analytics, also known as pick-rate/win-rate, win-delta / pick-rate (herein WD/PR) plots game material in a way that allows for easy comparison of how popular and powerful they are. WD/PR is commonly used as the basis for balance changes, however it is flawed. Whilst it does give a good account of material which should be tweaked to bring it more in line with the rest, it does not account for personal preference or represent mismatches between specific material. 

\begin{figure}
    \centering
    \includegraphics{pick-win.png}
    \caption{WD/PR generated 14-4, 10 days after initial release. Guidelines are drawn at x=0.125 or 1/8 as that is what would be the value if each were chosen equally, and at y = 0\% as that would be the value of material that wins as often as it loses.}
    \label{fig:WD/PR}
\end{figure}

% Lad, I can't get this warning to buzz off.

~\cref{fig:WD/PR}~shows that the Healer is both the most successful material and the least popular. \href{'https://staticctf.akamaized.net/J3yJr34U2pZ2Ieem48Dwy9uqj5PNUQTn/2MXEODt9QteOU3WGh4vmNE/51261bebfc558521b7411ccb78ece4fb/Y5S1_Matrix_Attackers.png'}{There are examples of this peculiar occurrence in other games.} Perhaps the low popularity makes the win-delta value more volatile, or the players are choosing based on factors other than their effectiveness. However popularity should be balanced for just as effectiveness is, as game designer Jeff Kaplan states, \textit{"The perception of balance is more powerful than balance itself."}

As our previous work pertains to metagame development, we wanted to investigate if we could use WD/PR variants to measure this development over time. To do this we have plotted WD/PR for every day since release. \WilliamToDo{Must decide what to do with the figure here. Probably just two days one after the other.}

\begin{figure}
    \centering
    \includegraphics{}
    \caption{WD/PR, over days}
    \label{fig:my_label}
\end{figure}

As a measure of how \textit{settled} the metagame is, we plotted each character on their own axes with their WD/PR values over days. Between each point there is a vector giving the disparity between points, the average of these vectors between each day should give a good indication of how \textit{settled} the metagame is.

\begin{figure}
    \centering
    \includegraphics{}
    \caption{WD/PR, per character, over days}
    \label{fig:my_label}
\end{figure}

\subsection{Elo and non-skill factors on placement}

Elo is a standard mechanism for measuring player skill, devised for Chess. \WilliamToDo{cite, cite, cite, here's how we did it for those interested.}

This is a far better measure of player skill than the \textit{friendly} measure of skill-points including in the game.

\begin{figure}
    \centering
    \includegraphics{}
    \caption{Elo of the playerbase plotted against luck}
    \label{fig:my_label}
\end{figure}

\WilliamToDo{discuss first-move bias...}

\begin{figure}
    \centering
    \includegraphics{}
    \caption{Elo of the playerbase plotted against going first}
    \label{fig:my_label}
\end{figure}

\subsection{Skill}

\WilliamToDo{We can do better than Elo, we may redo it, but also we can account for every correct move and mistake. Furthermore we can measure the cost of each mistake, \textbf{this is the cool novel bit.}}

\begin{figure}
    \centering
    \includegraphics{}
    \caption{Elo of the playerbase plotted against skill}
    \label{fig:my_label}
\end{figure}

\newpage



\end{document}

